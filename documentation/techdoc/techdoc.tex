\documentclass{article}
\usepackage{cite}
\usepackage{courier}
\usepackage[toc,page]{appendix}

\newcommand*{\createTitlePage}{\begingroup
\centering
\vspace*{6\baselineskip}

{\Huge WebAPI ECA-Engine}\\[4\baselineskip]

{\LARGE Technical Documentation} \\[\baselineskip]

\vspace*{25\baselineskip}

{Dominic Bosch\par}
{\itshape University of Basel\par}




\endgroup}


\begin{document}

\clearpage\createTitlePage
\thispagestyle{empty}

\newpage
\tableofcontents
\newpage


\section{Introduction}
\subsection{section}
\subsubsection{subsection}
t.b.d.
Introduction
%Documentation can be found on localhost:[http_port]/doc/

% TODO
% Key files within the application. This may include files created by the development team, databases accessed during the program's operation, and third-party utility programs.
% Functions and subroutines. This includes an explanation of what each function or subroutine does, including its range of input values and output values.
% Program variables and constants, and how they're used in the application.
% The overall program structure. For a disc-based application, this may mean describing the program's individual modules and libraries, while for a Web application, this may mean describing which pages use which files.

\section{Prerequisites}
Redis or write own DB Interface

Node.js

\section{Installation}
Crossplatform

\section{Configuration}

\section{Application Architecture}
The application is started through the webapi-eca module, which loads other modules such as the logging module, the configuration file handler, the persistence interface, the listener to HTTP requests and finally the ECA engine.
% TODO Architecture picture goes here!

\subsection{Modules}
\subsubsection{Webapi-ECA}
starting point
reads cli arguments
Initializes:
config
engine
persistence
http listener
logging
forks the event poller and sends him information about new rules so he can fetch the appropriate event poller modules.

\subsubsection{Persistence}
The persistence module is an interface to a persistent storage.
It stores the events in a queue, action invoker modules, event poller modules, rules, users and roles.
Event Queue
\texttt{event\_queue} (List): The event queue for all incoming events to be processed by the engine.
Action Invokers
\texttt{action-invokers} (Set of [aiId] keys): A set of all existing action invokers.
\texttt{action-invoker:[aiId]} (String): A stringified action invoker.
\texttt{action-params} (Set of [aiId]:[userId] keys): All existing action invoker parameters associated with a user.
\texttt{action-params:[aiId]:[userId]} (String): A stringified parameter object associated to an action invoker and a user.
Event Pollers
\texttt{event-pollers} (Set of [epId] keys): A set of all existing event pollers.
\texttt{event-poller:[epId]} (String): A stringified event poller.
\texttt{event-params} (Set of [epId]:[userId] keys): All existing event poller parameters associated with a user.
\texttt{event-params:[epId]:[userId]} (String): A stringified parameter object associated to an event poller and a user.
Rules
\texttt{rules} (Set of [ruleId] keys): A set of all existing rules.
\texttt{rule:[ruleId]:users} (Set of [userId] keys): Associated users to a rule.
\texttt{rule:[ruleId]} (String): Stringified rule object.
\texttt{rule:[ruleId]:active-users} (Set of [userId] keys): Users that have this rule activated.
Users
\texttt{users} (Set of [userId] keys): A set of all existing users.
\texttt{user:[userId]} (Hashmap): The flat user object.
\texttt{user:[userId]:rules} (Set of [ruleId] keys): Associated rules to a user.
\texttt{user:[userId]:active-rules} (Set of [ruleId] keys): Active rules.
\texttt{user:[userId]:roles} (Set of [roleId] keys): All roles a certain user is associated with.
Roles
\texttt{roles} (Set of [roleId] keys): A set of all existing roles.
\texttt{role:[roleId]:users} (Set of [userId] keys): All users associated to this role.

\subsection{Views/Webpages}
user interfaces
login
credentials entered in a login form are encrypted using a random key and only then sent to the server.
we fetch the google crypto-js module in the browser from
% \<script src="http://crypto-js.googlecode.com/svn/tags/3.1.2/build/rollups/sha3.js"></script>

and the engine fetches the same but modularized code from the npm repository via the package manager. Now we have the same crypto-js code in both modules

this also allows us to send privately stored modules and rules encrypted to the user, which will then see it decrypted after it arrived at the browser




\bibliography{user-manual}
\bibliographystyle{beast}

\newpage
\renewcommand*\appendixpagename{APPENDIX}
\renewcommand*\appendixtocname{APPENDIX}
\begin{appendices}
 \section{Things}
   \subsection{Important things}
   % \subsubsection{ecaserver.js}
 Some appendix content
 
\end{appendices}

\end{document}